\documentclass[11pt,twoside]{jreport}
\usepackage{thesisj}
\usepackage{lingmacros,jtheapa}

\title{英語起源日本語説}
\author{吉原源三郎}
\submitdate{2008年1月}
\copyrightyear{2008}

\principaladvisor{清水義範}
\firstreader{筒井康隆}
\secondreader{小松左京}

%%%% choose one %%%%
\bachelorthesistrue
%\mathesistrue
%\phddissertationtrue

%%%% choose any %%%%
\copyrighttrue
\committeetrue
%\figurespagetrue
%\tablespagetrue
%%%%%%%%%%%%%%%%

\begin{document}

\beforepreface

\prefacesection{謝辞}

この(従来の常識からみれば)非常識な、起爆性を秘めた論文を出版して下さっ
た風狂書房の大月小波氏に心から感謝の辞を贈りたい。また、本論文の成立に
並々ならぬご尽力をいただいた前田岩男氏、泡中夢之助氏には心から謝意を表
するものである。

\afterpreface

\chapter{はじめに}

  ここに発表する機会を得た本論文「英語起源日本語説」は、私の4年間にわ
たる研究の成果を、ひとまず体系的に構成し展開した最初の試みであり、今後
も更に論を深めていきたいと考えてはいるが、とりあえず学界に向けてはなつ
私の第一の矢、という性質を有している。

  はなたれた矢、という表現が決して大袈裟なものではないことは、これから
この内容に目を通し、読み終えた段階で誰の目にも理解できることであろう。
これはまさに過去の常識を根底から覆す衝撃力をもった全く独創的な論文であ
ると言うしかないものなのだから。

  しかしながら、はなたれた矢を正しく受け止める的があるかどうかについて
は、私は残念ながら悲観的な予測を持たざるを得ない。すなわち、この矢が本
来正鵠を射ていることに関しては疑いの余地がないにもかかわらず、旧弊で閉
鎖的な学界は的を外してひたすら逃げまわるであろうことが想像されるのであ
る。

  彼らは門外漢の新説などには耳を貸そうとせず、耳を貸したとしても黙殺す
るであろう。なぜなら、その説を受け入れることは彼らの依って立つ常識とい
うものの崩壊を認めることにつながるからである。どうして権威の柱の陰で細々
と生きる彼らがそんなことをするだろうか。

  そういうわけで、ここに私が展開しようとしている学説はその誕生の時から
迫害される運命にある。よろしい。その迫害を受け止めようではないか。ガリ
レオ・ガリレイと同じように、私もまた、それでも真実を述べずにはいられな
いのだから。\footnote{
%--------
本論文は15年前に書かれた原稿に手を入れ、現在のことろこれ以上直すとこ
ろがない、という形にまとめあげたものである。具体的に言えば、文章を今日
の読者にも理解しやすいように一部易しく書き直し、論証部分には、新たに34
の例証を加えた。
} %--------

\chapter{先行研究}

\section{英語の起源}

私がここで論証しようとしていることは次の短い一文に要約できる。

\enumsentence{
英語の起源は日本語である。
}

  しかし、内容の大きさが文の短さとは比例しないことは、言を俟たない。思
えば、従来どの比較言語学者も、日本語が何か他の言語の起源であるというよ
うな発想を持ち得なかったのである。

  中学一年の時だったと思う。試験の時、次の言葉を英語にせよ、という問題
が出て、その中に<名前>というのがあった。

  私の隣の席の少年は、その<名前>を意味する英単語が思い出せなかった。
それで彼は、とにかく何か書いておこうという気持から、答えの欄にローマ字
で namae と書いた。あとで答案用紙を返してもらう時、彼は悪ふざけをする
なと先生に叱られるかも知れないと心配したが、その心配は無用だった。彼の
答の namae は、正解と比べて a がひとつ多かっただけで、先生は彼が苦肉の
策としてローマ字を書いたということを見抜けなかったのである。

\enumsentence{
name (ネーム)と namae (名前)
}
\noindent
この時彼の言った言葉、「英語では名前を名めー、というのか」を思い出した
時、私の頭に電撃的にひらめくものがあった。

  name と namae の類似性は、偶然というにはあまりにも近すぎる。name の
語源は namae ではないのか、と想像せざるを得ないではないか。

  研究に値する発見であるように、思考の柔軟な私には思えた。これが私の学
説の出発点となったのである。

  日英語単語間に、ほかに類似性の見られるものはないだろうかという探索の
作業が積み重ねられた。その結果、ひとつの新学説を成すに十分と思われる
292例もが発見されたのである。\footnote{
%--------
先の脚注で触れたように、その後の研究の進展により、英語の起源が日本語
であることを立証する例は326になった。
} %--------

  ここにその中から2, 3 の例を引けば次のようになる。

\eenumsentence{
\item 汁 (ju) → juice (汁)
\item 斬る (kiru) → kill (殺す)
\item  だるい (durui) → dull (鈍い)
}

  これだけの例を見ただけでも、日本語と英語との間には否定し難い関連があ
ることが感じられるであろう。ましてや、292例のすべてに目を通し終えた時、
思考の柔軟な読者には、英語と日本語には何か根本的な共通性があることを確
信せざるを得ないであろう。\footnote{
%--------
もちろん、次のような反論が即座に返ってくることは予想できる。

\eenumsentence[i]{
\item 文法が違う。
\item 日本語は膠着語で英語は屈折語である。
\item いつ、どのように伝わったのかはっきりしない。
}
\noindent
これらの問題は将来の研究の進展に委ねたいが、本能的に二つの言語が兄弟で
あることは直感できるはずである。常識という思考力の枷に毒された三流学者
でない限りは。
} %--------

\section{従来の研究の問題点}

  日本語と他の言語の共通性を指摘する研究はもちろん私の研究が最初のもの
ではない。古くは、明治時代に木村鷹太郎氏が、日本人の先祖はエジプト人で
あり、ギリシャ人でありローマ人であるという大胆な説を主張し日本話とギリ
シャ語・ラテン話・英語との類似を指摘している \cite{木村81}。木村によると
次のような英語の単語は日本語と起源が同じだという。

\enumsentence{
\begin{tabular}[t]{*{9}{l|}l}
    夕べ&ダメ  &君  &籠  &なんぼ&潮 &骨  &ソロリ&身&百合\\
\hline
    eve &damage&king&cage&number&see&bone&slowly&me&lily
\end{tabular}
}
\noindent
一目瞭然であるが、これは単なる英単語の駄洒落による暗記法であり、学問的
な裏付けを欠いている。

  最近に至るまで、このような胡散臭い研究は後を断たないが、英語圏からの
研究もある、\citeA{スミサナ92} によると、日本人の先祖はアメリカ・インディ
アンであり、インディアン語に由来するアメリカの地名・人名・部族名はすべ
て日本語で解読できるという。例えば、次のような例が「証拠」としてあげら
れている。

\enumsentence{
\addtolength{\tabcolsep}{-0.3em}
\begin{tabular}[t]{@{}l|l||l|l||l|l}
テキサス & 敵刺す & ミズーリ & 水入り江 & マサチューセッツ & 鱒駐節\\
ミシシッピー & 水疾飛 & ワイオミング & 上の民家 & オクラホマ & 遅れ本真\\
オハイオ & おはよう & カンザス & 関西 & ケンタッキー & 関東京\\
メキシコ & 茅始処 & カナダ & 金田 & ナイアガラ & 荷揚げ場\\
アパッチ & あっぱれな者 & エスキモー & アシカの肝 & ジェロニモ & 地浪人者
\end{tabular}
}
\noindent
これではまるで暴走族の名ではないか。訳者あとがきによれば、著者は「最近
の日英辞典」と使ったとあるが、ケンタッキーは明治時代になってからできた
のであろうか。

  他にも、\citeA{安田55}、\citeA{藤村89}、\citeA{李89} などの、万葉集
を根拠にして日本語をレプチャ語や朝鮮語と結び付ける邪説、\citeA{吉田91} 
のような、日本語をあらゆるものの起源にしてしまう暴説などがあとを断たな
いが、これらに対しては\citeA{安本91} による学問的な見地からの反論にゆ
ずるとして、本論文で私が主張することはこれらの先行非研究とは一線を画す
ものであることを強調しておきたい。その意味では、英語の起源を日本語に置
く真に学問的な先行研究は存在しないと言っても過言ではない。

\chapter{語呂あわせとの相違}

本稿の初期の版に対して、次のような批判がよせられたことがある。

\begin{quote}
    吉原源三郎とかいう素人のたわごと「英語起源日本語説』なるものは、
    犬小屋をケンネル (kennel)、辞書を字引く書なり (dictionary) と言
    うたぐいの与太にすぎない \cite{山崎91a}。
\end{quote}

  一体彼はどうして私の説を、犬小屋をケンネル(犬寝る)と覚えるような語
呂あわせと同じものだと思ってしまったのであろう。試みに私が例示した、日
本語から英語になったと思われる単語のいくつかを引出してみれば、すぐさま
それが語呂あわせの英単語記憶法などとは全く違うものであることが理解でき
るではないか。たとえば、

\eenumsentence{
\item 坊や (boya) → boy (少年)
\item 名前 (namae) → name (名前)
}
\noindent
これらのどこが字引く書なり、と同じだというのであろう。

  boya なる単語が、母音の弱化により boy に変じていくという仮説は、山崎
氏がその代表的著作\citeA{山崎69} の中で展開した学説と全く同じではないか。

  だから、氏が私の論理を与太だと決めつけることは、そのまま自分の論理を
否定することになるのである。彼には自分が何を言っているのかわかっていな
いのだろうか。

  いずれにしても、山崎氏の私への反論は、私のあげた例証のうち少々言語伝
達の構造が複雑なものについて集中していることは明らかである。

  そのくせ、氏は私のあげる例のうち、あまりにも単純なもの、ストレートに
意味が伝達されているもののことについては口をふさいで語ろうとしない。す
なわち、反論の余地がないのである。次の例を参照されたい。

\enumsentence{
\begin{tabular}[t]{lcl}
    負う (ou)       & → & owe (負う)\\
    たぐる (taguru) & → & tag(引き寄せる)\\
    疾苦 (sikku)    & → & sick(病気)
\end{tabular}
}
\noindent
これらの例は、ケチのつけようがないほど完全に言語伝達されている。そこで
山崎氏はこういう沢山の例を無視するわけである。

  それならば、次の例についてどうして氏は反論しないのであろう。これはな
かなか伝達の構造が複雑なのであるのに。

\enumsentence{
場取る (batoru) → battle (戦い)
}
\noindent
この例は、日本語が英語のもとになったことを見事に証明してくれるものなの
である。

  すなわち、欧米では戦いとは、敵の王を殺すことであったのに対し、日本で
は、敵の領地を奪うこと、つまり場を取る(盗る)ことであったという文化的
差異がこの一語にはこめられている。

  さらに、山崎氏は、次のような反論をしている。

\begin{quote}
    この吉原源三郎説の最も弱いところは、比較される二つの言葉に意味
    の共通性がないことが非常に多いことである。たとえば彼は、日本語
    の(掘る)という動詞を、hole (穴)という英語の名詞と対比して
    みせる。掘るから穴だというこの漫才のような対比は、学問的立場か
    らは到底認められないものである。cold (冷たい)を(凍るぞ)と
    やってみせるのも同様のおちゃらけである \cite[96]{山崎91b}。
\end{quote}
\noindent
彼の頭の中にある言語の伝達は、二国間でひとつの言葉が全く同じ意味に伝わ
る、ということであるらしい。それ以外のものは、たとえどんなに共通性が強
くても、無いのと同じ、なのである。

  英語の hole が、掘る、という意味ならば認めるが、そうでない以上そんな
ものはおちゃらけだ、と言い張る彼は、本当に言語学者なのだろうか。言語と
いうものが、いかに誤って伝達され、時と共にその意味すら変化していくとい
うことを知らない言語学者がどうして存在しているのか、私には謎である。

  たとえば、「前」という、位置、方向を表す言葉が、「お前」となって二人
称の人物を表す言葉になり、更に下ってそれが尊称から蔑称に変化した、とい
うような事実は、言語学者山崎にとっては、無いのと同じ、なのであろうか。
もしそうだとしたらそんな学者を認めておくわけにはいかない。

  私の言うところの(伝日本語人)が、地面に穴をあけてこと作業を「掘る」
というのだと教えた時、(源英語人)がその言葉の意味を、そこにあいた穴だ
と解釈したというのは非常にありうべき事柄である。そこに漫才師の登場する
余地はほとんどない。

  同様に、「この寒さでは、明日は水が凍るだろう」と言った時、k\^oru → 
cold、コールドが冷たいという意味に受け止められたことは誰のおちゃらけで
もない。同じような考察によって、私は堂々と次のような類似の例も提出して
いるのである。

\eenumsentence{
\item 抛る (h\^oru) → fall (落ちる)
\item 述べる (noberu) → novel (小説)
}

\chapter{おわりに}

本来は、どのようにして日本語が海を渡り、英語の起源になったかに関する仮
説を本論文に含める予定であったが、この問題に関して本格的に論じるには別
の大部の書を必要とするし、その作業は既に私の手によってなされているため
に、割愛せざるを得なかった(拙著\citeA{吉原93} を参照されたい)。


\bibliographystyle{jtheapa}
\bibliography{thesis}

\appendix
\chapter{おまけ}
おまけ

\end{document}
