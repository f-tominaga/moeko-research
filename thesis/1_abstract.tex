\chapter{はじめに}
%章番号が不要な場合は『\chapter*{概要}』とする
\section{ロボットの知能化}
ロボット元年といわれている19800年から現在に至るまで,各分野において様々なロボットが研究,開発されている.これまでの多くのロボットは,工場,核廃棄物処理場,災害現場,宇宙,深海などの極限環境での活躍を期待して開発されてきた.一方で,現在は人の生活をサポートする警備や福祉,医療ロボットの廻達も行われており,人との協調をテーマとしてサービス分野への進出が期待されている.日常生活では見えないところで働いてきたロボットが人間の目にむえる環境へと活躍の場が広がってきた.
最近ではペットロボット,二足歩行ロボット,家電ロボット等が一般向けに販売さえれるようになってきた.ロボットによるサッカーの大会"RoboCup"等のロボット大会,ものづくりの大会ロボマス等の影響は大きい.今後もロボットの活躍の場は広がり,ロボットを利用した新しいアプリケーションが出てくるであろう.


  ロボットへの社会的期待は, 持続可能な社会の実現, 急速な少子高齢化による働き手不足の補填, 産業基盤の再構築など多岐に渡り, 日常的な接触を伴うロボットの必要性も高まっている. 安全安心なロボット社会を実現するためには, 社会的期待と研究開発の方向性の違いを避け, 理解しやすい方法で研究成果を導き出すことが必要である.  したがって, ロボットとの共存をどのように達成するか, そして共生社会をどのように実現するべきか議論することが不可欠になる. 人間とロボットの共生社会では, 人間とロボットは相互に作用し合い, 相互の理解ができるので, 彼ら自身の行動だけでなく, 他のすべての自律エージェント(環境を感知し,自律的方針に従い行動する個体)とも相互作用することができるのが理想です. そのため本研究の目的としては人間と環境を共有することができる知能ロボットにおける適切なアルゴリズムを開発することである. 本研究ではそのテストベットとして限定された空間であるサッカーというフィールドを用いる. サッカーは戦略, 選手間の協力, 予測不可能な動き, そして共通の目標を含んでいるため, 以上に述べたような知能ロボットの行動アルゴリズムに類似する点が多く, 開発に最適なテストベッドである. 
  \\ これまでに,人間によるサッカーにおける戦略行動の解析や,ロボットによるサッカーとの比較等を行い,人間とロボットにおけるサッカーの協調行動の相違点に着目して自己組織化マップ(SOM)での解析を行った.\cite{Moeko}
\\ 結果として,まず人間によるサッカーの試合では,チーム内でのボールとの距離の長短による順位が選手のポジショニングに影響を与えていることがわかった.またこの結果は,普段から同じメンバーで試合をしている玄人にのみ現れ,初めてチームを結成して行った素人チームでは発現が見られなかった.このことは,ボールとの距離により個人がチームとしてのポジショニングをしていることを示唆している.
\\ また,人間とロボットのサッカー試合の比較では,例外的動きをするゴールキーパーを除外した,敵味方全選手の位置とボールの位置を入力とし,シュート行動まで場面を解析した.結果としては,人間,ロボットどちらによる試合でも,オフェンス行動をおこなったときとディフェンス行動を行ったときとで,ことなるチームフォーメーションをしていることがわかった.加えて,人間の試合の方がロボットの試合より,細かく場面分けがなされていた.(チームとして,最初から最後までディフェンスであったときと,ディフェンスからオフェンスに変化するチームフォーメーションが区別されていた.)このことは,人間,ロボットどちらの試合でも戦略的行動選択がなされているが,人間の試合では,ディフェンスフォーメーションとオフェンスフォーメーションの間に移行フォーメーションが存在することが示唆される.
\\ このように,チームとして複数エージェントで行われるサッカーの試合では,必ず協調のための位置取りが存在し,あるチームのある選手の行動選択には周りの選手(敵味方含む)の位置取りやチームとしてのフォーメーションが関係していることが明らかとなった.
\\ 本研究では,以上の研究結果を踏まえて,サッカーロボットのための行動学習システムを開発する.この行動学習システムでは,あるチーム内のゴールキーパーを除いた全選手の行動を学習する.評価として,学習した行動選択器が,どれほどの再現性を有するかを指標とする.このシステムにより,あるチームにおける全選手の行動選択アルゴリズムを再現することができ,ロボットは試合中に得た情報から瞬時に次の協調的な行動を選択することが可能となる.例えば,小学生で構成されたチームの試合データを学習すると,ロボットはその小学生に合わせた行動基準で次の行動を選択し,ヨーロッパのクラブチームのサッカー試合データを学習すると,そのチーム特有のフォーメーションで戦略的な試合展開をすることが可能となる.
\\ このように,あるエージェントの集合体の行動選択アルゴリズムの学習が可能となると,ロボットだけで構成させたチームのみではなく,人間とロボットで構成されるような異種エージェント集合体で相互に共通の「常識的なルール」を共有することが可能となる.ひいては,これからの人間ロボット共生社会にとって大きな役割を担うこととなる.
\\ 学習には,テンソル自己組織化マップ(TSOM)\cite{Kohonen}を用いる. 1.1節, 1.2節, および1.3節では, このような学習システム構築を目的として近年研究されてきた学習アルゴリズムについての概要を提示する. 加えて,SOMのアルゴリズムと実験方法については2章でより詳しく説明する. 3章で実験と結果を示し議論する. 
\subsection{協調行動}%-----------
協調行動は, 異なる自律エージェントが共通のタスクを実行しながらコミュニケーションを図るときに重要な側面である. 多くの場合,単一のエージェントでのタスクを実行は必ずしも効率的とは言えず, ここ数年で, 困難な問題を解決するためにマルチエージェントシステム(MAS)を研究している研究者も多い. マルチエージェントシステムは, 複数のエージェント(自律エージェント)が協調行動によって共通の目標を達成しようとするシステムである .  センサーから取得したデータに基づいてリアルタイムで決定を下すことによって, エージェント同士に加え,環境と相互作用する. \cite{PS}\cite{KU}
MASのテストベッドとして有名な,RoboCupは,世界的ロボット大会であり,強化学習やニューラルネットワークなどの学習方法を使用してマルチエージェントの協調を一つの課題としている.  RoboCupはランドマークプロジェクトであり,2050年までに,人間のサッカーワールドカップチャンピオンチームに勝つロボットチームの制作プロジェクトである\cite{BS}. 
例としてこれまでに,Sandholm and Crites\cite{TWS}によれば, 十分な測定データと行動が利用可能であれば, 強化学習は反復囚人のジレンマに対して最適な手法であると示された. さらに, Arai\cite{SA}は, 環境がグリッドとしてモデル化されている場合のマルチエージェントシステムにおける追跡問題に関するQラーニングと利益共有の方法を比較し, 協力的な行動が利益共有の間で明確に現れることを示した. しかしながら, これらの研究は, 実環境で動作するロボットへの応用をまだ考慮されていない. 
\subsection{ロボット学習}%-----------
学習アルゴリズムの中で,教師なし学習はMASシステムにとって有望な方法である.教師なし学習の利点は,ロボットが環境に関する以前の情報やロボット自体の前提知識を持っている必要がないことである. しかし,学習には多くのテストや経験が必要であるため, 設計者は, 状態空間とアクション空間を定義するために, 試合中に考慮すべき重要なパラメータと変数を明確に定義することがコストと時間を削減するために重要である. 状態空間の設計は, ロボットが行動空間で何ができるかによって異なる. 同時に, 状態空間はロボットの能力とそれ自身の行動空間に従って定義される. 2つの空間は互いに相互接続している. 効率的な方法で状態空間を設計するために, Asada\cite{YT}は, 状態空間を最初に2つの主な状態に分割し, 次に状態数を増やしながら再帰的に多数の層に分割する方法を提案した. しかし, いくらかの偏りの問題は依然としてロボットの行動に影響を及ぼし, 何らかの誤った行動および習慣を引き起こす可能性があると示された.
\subsection{人間ロボット協調}%-----------
上述したように,人間 - ロボット共生社会は多くの研究者にとって関心のあるトピックであり,いくつかの研究が行われてきた.したがって,そのような共生システムでは, ロボットが人間の行動を理解し解釈し, それに応じて行動することが必須である.しかし,これまでの研究では目標として,ロボットの行動のみ,または特定の数の人間とロボットの間で正確に定義されたインタラクションに基づく行動に関することが大半であった. また, マルチエージェントシステムでは,人間とロボットとの間の通信は,通常,ある種のインタフェース,例えばコマンド音声またはジェスチャによって行われている.しかし,システムが非常に複雑な場合は特に,これでは不十分な場合が存在する.したがって,次のステップとしてロボットに,人間がそうであるように,いくつかの状況を理解し,その挙動に予測的に適応させることを目指す. この研究の主な目的は, 人間同士の協調行動を考え,それをロボットに応用することである.
\\ニューラルネットワークへの入力ベクトルの要素は,人間とロボットの行動との間の可能な最小のギャップを達成するように選択される.そのためには, まず人間の行動を理解し,そのような行動を理解して模倣できるロボットを開発することが重要でである.文献\cite{MM}では,フットサルゲームにおける人間とロボットの行動の研究に焦点を当てている. なぜなら, これは動的な環境,いくつかの制約を伴う優れたテストベッドであり,リアルタイムの計画が必要だからである. これらは, ロボットが将来動作する可能性がある一般的な共生システムの特性となる. 使用されるアルゴリズムでは,スコア,コーナーキック,ペナルティキック,ボールを持っているチームなど, ゲーム内の特定のイベントに応じて,SOMを使用してプレーヤーのポジションが分析された. その結果, 人間の試合もロボットの試合も同様に, 試合における場面を評価することが可能であり, 学習する入力要素を増やすことによって,環境と行動の写像関係を示すSOMの作成によってロボットによる自律的行動の発現が可能であることが判明した. この論文では,その写像関係を求め,テンソル自己組織化マップによりロボット同士の試合を学習し,サッカーロボットの行動学習システムの開発を行う.


  \footnote{
%--------
本論文は15年前に書かれた原稿に手を入れ、現在のことろこれ以上直すとこ
ろがない、という形にまとめあげたものである。具体的に言えば、文章を今日
の読者にも理解しやすいように一部易しく書き直し、論証部分には、新たに34
の例証を加えた。
} %--------


