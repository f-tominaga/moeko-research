\documentclass[11pt,twoside]{jreport}
\usepackage{thesisj}
\usepackage{lingmacros,jtheapa}

\title{TSOMを用いた行動戦略解析}
\author{富永萌子}
\submitdate{2021年3月31日}
\copyrightyear{2020}

\principaladvisor{石井和男}
\firstreader{我妻先生}
\secondreader{林英治先生}

%%%% choose one %%%%
%\bachelorthesistrue
%\mathesistrue
\phddissertationtrue

%%%% choose any %%%%
\copyrighttrue
\committeetrue
\figurespagetrue
\tablespagetrue
%%%%%%%%%%%%%%%%

\begin{document}

\beforepreface

\prefacesection{謝辞}
本研究を遂行するにあたり,熱心なご指導と有益なご助言を賜りますとともに,研究者としての心得を御教授してくださいました九州工業大学大学院生命体工学研究科人間知能システム工学専攻 石井和男教授に心から感謝します.




\afterpreface
\chapter*{Abstruct}
With the progress of technology, the realization of a symbiotic society with human beings and robots sharing the same environment has become an important subject. An example of this kind of systems is soccer game. Soccer is a multi-agent game that requires strategies by taking into account each member's position and actions. In this paper, we discuss the results of the development of a learning system that uses SOM to select behaviors depending on the situation.This system can reproduce the action selection algorithm of all players in a certain team, and the robot can instantly select the next cooperative action from the information obtained during the game.Because of this system, common sense rules was shared to learn an action selection algorithm for a set of agents, not only a team consisting of robots alone, but also a group of heterogeneous agents consisting of humans and robots.
\chapter{はじめに}
%章番号が不要な場合は『\chapter*{概要}』とする
\section{ロボットの知能化}
ロボット元年といわれている19800年から現在に至るまで,各分野において様々なロボットが研究,開発されている.これまでの多くのロボットは,工場,核廃棄物処理場,災害現場,宇宙,深海などの極限環境での活躍を期待して開発されてきた.一方で,現在は人の生活をサポートする警備や福祉,医療ロボットの廻達も行われており,人との協調をテーマとしてサービス分野への進出が期待されている.日常生活では見えないところで働いてきたロボットが人間の目にむえる環境へと活躍の場が広がってきた.
最近ではペットロボット,二足歩行ロボット,家電ロボット等が一般向けに販売さえれるようになってきた.ロボットによるサッカーの大会"RoboCup"等のロボット大会,ものづくりの大会ロボマス等の影響は大きい.今後もロボットの活躍の場は広がり,ロボットを利用した新しいアプリケーションが出てくるであろう.


  ロボットへの社会的期待は, 持続可能な社会の実現, 急速な少子高齢化による働き手不足の補填, 産業基盤の再構築など多岐に渡り, 日常的な接触を伴うロボットの必要性も高まっている. 安全安心なロボット社会を実現するためには, 社会的期待と研究開発の方向性の違いを避け, 理解しやすい方法で研究成果を導き出すことが必要である.  したがって, ロボットとの共存をどのように達成するか, そして共生社会をどのように実現するべきか議論することが不可欠になる. 人間とロボットの共生社会では, 人間とロボットは相互に作用し合い, 相互の理解ができるので, 彼ら自身の行動だけでなく, 他のすべての自律エージェント(環境を感知し,自律的方針に従い行動する個体)とも相互作用することができるのが理想です. そのため本研究の目的としては人間と環境を共有することができる知能ロボットにおける適切なアルゴリズムを開発することである. 本研究ではそのテストベットとして限定された空間であるサッカーというフィールドを用いる. サッカーは戦略, 選手間の協力, 予測不可能な動き, そして共通の目標を含んでいるため, 以上に述べたような知能ロボットの行動アルゴリズムに類似する点が多く, 開発に最適なテストベッドである. 
  \\ これまでに,人間によるサッカーにおける戦略行動の解析や,ロボットによるサッカーとの比較等を行い,人間とロボットにおけるサッカーの協調行動の相違点に着目して自己組織化マップ(SOM)での解析を行った.\cite{Moeko}
\\ 結果として,まず人間によるサッカーの試合では,チーム内でのボールとの距離の長短による順位が選手のポジショニングに影響を与えていることがわかった.またこの結果は,普段から同じメンバーで試合をしている玄人にのみ現れ,初めてチームを結成して行った素人チームでは発現が見られなかった.このことは,ボールとの距離により個人がチームとしてのポジショニングをしていることを示唆している.
\\ また,人間とロボットのサッカー試合の比較では,例外的動きをするゴールキーパーを除外した,敵味方全選手の位置とボールの位置を入力とし,シュート行動まで場面を解析した.結果としては,人間,ロボットどちらによる試合でも,オフェンス行動をおこなったときとディフェンス行動を行ったときとで,ことなるチームフォーメーションをしていることがわかった.加えて,人間の試合の方がロボットの試合より,細かく場面分けがなされていた.(チームとして,最初から最後までディフェンスであったときと,ディフェンスからオフェンスに変化するチームフォーメーションが区別されていた.)このことは,人間,ロボットどちらの試合でも戦略的行動選択がなされているが,人間の試合では,ディフェンスフォーメーションとオフェンスフォーメーションの間に移行フォーメーションが存在することが示唆される.
\\ このように,チームとして複数エージェントで行われるサッカーの試合では,必ず協調のための位置取りが存在し,あるチームのある選手の行動選択には周りの選手(敵味方含む)の位置取りやチームとしてのフォーメーションが関係していることが明らかとなった.
\\ 本研究では,以上の研究結果を踏まえて,サッカーロボットのための行動学習システムを開発する.この行動学習システムでは,あるチーム内のゴールキーパーを除いた全選手の行動を学習する.評価として,学習した行動選択器が,どれほどの再現性を有するかを指標とする.このシステムにより,あるチームにおける全選手の行動選択アルゴリズムを再現することができ,ロボットは試合中に得た情報から瞬時に次の協調的な行動を選択することが可能となる.例えば,小学生で構成されたチームの試合データを学習すると,ロボットはその小学生に合わせた行動基準で次の行動を選択し,ヨーロッパのクラブチームのサッカー試合データを学習すると,そのチーム特有のフォーメーションで戦略的な試合展開をすることが可能となる.
\\ このように,あるエージェントの集合体の行動選択アルゴリズムの学習が可能となると,ロボットだけで構成させたチームのみではなく,人間とロボットで構成されるような異種エージェント集合体で相互に共通の「常識的なルール」を共有することが可能となる.ひいては,これからの人間ロボット共生社会にとって大きな役割を担うこととなる.
\\ 学習には,テンソル自己組織化マップ(TSOM)\cite{Kohonen}を用いる. 1.1節, 1.2節, および1.3節では, このような学習システム構築を目的として近年研究されてきた学習アルゴリズムについての概要を提示する. 加えて,SOMのアルゴリズムと実験方法については2章でより詳しく説明する. 3章で実験と結果を示し議論する. 
\subsection{協調行動}%-----------
協調行動は, 異なる自律エージェントが共通のタスクを実行しながらコミュニケーションを図るときに重要な側面である. 多くの場合,単一のエージェントでのタスクを実行は必ずしも効率的とは言えず, ここ数年で, 困難な問題を解決するためにマルチエージェントシステム(MAS)を研究している研究者も多い. マルチエージェントシステムは, 複数のエージェント(自律エージェント)が協調行動によって共通の目標を達成しようとするシステムである .  センサーから取得したデータに基づいてリアルタイムで決定を下すことによって, エージェント同士に加え,環境と相互作用する. \cite{PS}\cite{KU}
MASのテストベッドとして有名な,RoboCupは,世界的ロボット大会であり,強化学習やニューラルネットワークなどの学習方法を使用してマルチエージェントの協調を一つの課題としている.  RoboCupはランドマークプロジェクトであり,2050年までに,人間のサッカーワールドカップチャンピオンチームに勝つロボットチームの制作プロジェクトである\cite{BS}. 
例としてこれまでに,Sandholm and Crites\cite{TWS}によれば, 十分な測定データと行動が利用可能であれば, 強化学習は反復囚人のジレンマに対して最適な手法であると示された. さらに, Arai\cite{SA}は, 環境がグリッドとしてモデル化されている場合のマルチエージェントシステムにおける追跡問題に関するQラーニングと利益共有の方法を比較し, 協力的な行動が利益共有の間で明確に現れることを示した. しかしながら, これらの研究は, 実環境で動作するロボットへの応用をまだ考慮されていない. 
\subsection{ロボット学習}%-----------
学習アルゴリズムの中で,教師なし学習はMASシステムにとって有望な方法である.教師なし学習の利点は,ロボットが環境に関する以前の情報やロボット自体の前提知識を持っている必要がないことである. しかし,学習には多くのテストや経験が必要であるため, 設計者は, 状態空間とアクション空間を定義するために, 試合中に考慮すべき重要なパラメータと変数を明確に定義することがコストと時間を削減するために重要である. 状態空間の設計は, ロボットが行動空間で何ができるかによって異なる. 同時に, 状態空間はロボットの能力とそれ自身の行動空間に従って定義される. 2つの空間は互いに相互接続している. 効率的な方法で状態空間を設計するために, Asada\cite{YT}は, 状態空間を最初に2つの主な状態に分割し, 次に状態数を増やしながら再帰的に多数の層に分割する方法を提案した. しかし, いくらかの偏りの問題は依然としてロボットの行動に影響を及ぼし, 何らかの誤った行動および習慣を引き起こす可能性があると示された.
\subsection{人間ロボット協調}%-----------
上述したように,人間 - ロボット共生社会は多くの研究者にとって関心のあるトピックであり,いくつかの研究が行われてきた.したがって,そのような共生システムでは, ロボットが人間の行動を理解し解釈し, それに応じて行動することが必須である.しかし,これまでの研究では目標として,ロボットの行動のみ,または特定の数の人間とロボットの間で正確に定義されたインタラクションに基づく行動に関することが大半であった. また, マルチエージェントシステムでは,人間とロボットとの間の通信は,通常,ある種のインタフェース,例えばコマンド音声またはジェスチャによって行われている.しかし,システムが非常に複雑な場合は特に,これでは不十分な場合が存在する.したがって,次のステップとしてロボットに,人間がそうであるように,いくつかの状況を理解し,その挙動に予測的に適応させることを目指す. この研究の主な目的は, 人間同士の協調行動を考え,それをロボットに応用することである.
\\ニューラルネットワークへの入力ベクトルの要素は,人間とロボットの行動との間の可能な最小のギャップを達成するように選択される.そのためには, まず人間の行動を理解し,そのような行動を理解して模倣できるロボットを開発することが重要でである.文献\cite{MM}では,フットサルゲームにおける人間とロボットの行動の研究に焦点を当てている. なぜなら, これは動的な環境,いくつかの制約を伴う優れたテストベッドであり,リアルタイムの計画が必要だからである. これらは, ロボットが将来動作する可能性がある一般的な共生システムの特性となる. 使用されるアルゴリズムでは,スコア,コーナーキック,ペナルティキック,ボールを持っているチームなど, ゲーム内の特定のイベントに応じて,SOMを使用してプレーヤーのポジションが分析された. その結果, 人間の試合もロボットの試合も同様に, 試合における場面を評価することが可能であり, 学習する入力要素を増やすことによって,環境と行動の写像関係を示すSOMの作成によってロボットによる自律的行動の発現が可能であることが判明した. この論文では,その写像関係を求め,テンソル自己組織化マップによりロボット同士の試合を学習し,サッカーロボットの行動学習システムの開発を行う.


  \footnote{
%--------
本論文は15年前に書かれた原稿に手を入れ、現在のことろこれ以上直すとこ
ろがない、という形にまとめあげたものである。具体的に言えば、文章を今日
の読者にも理解しやすいように一部易しく書き直し、論証部分には、新たに34
の例証を加えた。
} %--------



\clearpage
\chapter{先行研究}

\section{英語の起源}

私がここで論証しようとしていることは次の短い一文に要約できる。

\enumsentence{
英語の起源は日本語である。
}

  しかし、内容の大きさが文の短さとは比例しないことは、言を俟たない。思
えば、従来どの比較言語学者も、日本語が何か他の言語の起源であるというよ
うな発想を持ち得なかったのである。

  中学一年の時だったと思う。試験の時、次の言葉を英語にせよ、という問題
が出て、その中に<名前>というのがあった。

  私の隣の席の少年は、その<名前>を意味する英単語が思い出せなかった。
それで彼は、とにかく何か書いておこうという気持から、答えの欄にローマ字
で namae と書いた。あとで答案用紙を返してもらう時、彼は悪ふざけをする
なと先生に叱られるかも知れないと心配したが、その心配は無用だった。彼の
答の namae は、正解と比べて a がひとつ多かっただけで、先生は彼が苦肉の
策としてローマ字を書いたということを見抜けなかったのである。

\enumsentence{
name (ネーム)と namae (名前)
}
\noindent
この時彼の言った言葉、「英語では名前を名めー、というのか」を思い出した
時、私の頭に電撃的にひらめくものがあった。

  name と namae の類似性は、偶然というにはあまりにも近すぎる。name の
語源は namae ではないのか、と想像せざるを得ないではないか。

  研究に値する発見であるように、思考の柔軟な私には思えた。これが私の学
説の出発点となったのである。

  日英語単語間に、ほかに類似性の見られるものはないだろうかという探索の
作業が積み重ねられた。その結果、ひとつの新学説を成すに十分と思われる
292例もが発見されたのである。\footnote{
%--------
先の脚注で触れたように、その後の研究の進展により、英語の起源が日本語
であることを立証する例は326になった。
} %--------

  ここにその中から2, 3 の例を引けば次のようになる。

\eenumsentence{
\item 汁 (ju) → juice (汁)
\item 斬る (kiru) → kill (殺す)
\item  だるい (durui) → dull (鈍い)
}

  これだけの例を見ただけでも、日本語と英語との間には否定し難い関連があ
ることが感じられるであろう。ましてや、292例のすべてに目を通し終えた時、
思考の柔軟な読者には、英語と日本語には何か根本的な共通性があることを確
信せざるを得ないであろう。\footnote{
%--------
もちろん、次のような反論が即座に返ってくることは予想できる。

\eenumsentence[i]{
\item 文法が違う。
\item 日本語は膠着語で英語は屈折語である。
\item いつ、どのように伝わったのかはっきりしない。
}
\noindent
これらの問題は将来の研究の進展に委ねたいが、本能的に二つの言語が兄弟で
あることは直感できるはずである。常識という思考力の枷に毒された三流学者
でない限りは。
} %--------

\section{従来の研究の問題点}

  日本語と他の言語の共通性を指摘する研究はもちろん私の研究が最初のもの
ではない。古くは、明治時代に木村鷹太郎氏が、日本人の先祖はエジプト人で
あり、ギリシャ人でありローマ人であるという大胆な説を主張し日本話とギリ
シャ語・ラテン話・英語との類似を指摘している \cite{木村81}。木村によると
次のような英語の単語は日本語と起源が同じだという。

\enumsentence{
\begin{tabular}[t]{*{9}{l|}l}
    夕べ&ダメ  &君  &籠  &なんぼ&潮 &骨  &ソロリ&身&百合\\
\hline
    eve &damage&king&cage&number&see&bone&slowly&me&lily
\end{tabular}
}
\noindent
一目瞭然であるが、これは単なる英単語の駄洒落による暗記法であり、学問的
な裏付けを欠いている。

  最近に至るまで、このような胡散臭い研究は後を断たないが、英語圏からの
研究もある、\citeA{スミサナ92} によると、日本人の先祖はアメリカ・インディ
アンであり、インディアン語に由来するアメリカの地名・人名・部族名はすべ
て日本語で解読できるという。例えば、次のような例が「証拠」としてあげら
れている。

\enumsentence{
\addtolength{\tabcolsep}{-0.3em}
\begin{tabular}[t]{@{}l|l||l|l||l|l}
テキサス & 敵刺す & ミズーリ & 水入り江 & マサチューセッツ & 鱒駐節\\
ミシシッピー & 水疾飛 & ワイオミング & 上の民家 & オクラホマ & 遅れ本真\\
オハイオ & おはよう & カンザス & 関西 & ケンタッキー & 関東京\\
メキシコ & 茅始処 & カナダ & 金田 & ナイアガラ & 荷揚げ場\\
アパッチ & あっぱれな者 & エスキモー & アシカの肝 & ジェロニモ & 地浪人者
\end{tabular}
}
\noindent
これではまるで暴走族の名ではないか。訳者あとがきによれば、著者は「最近
の日英辞典」と使ったとあるが、ケンタッキーは明治時代になってからできた
のであろうか。

  他にも、\citeA{安田55}、\citeA{藤村89}、\citeA{李89} などの、万葉集
を根拠にして日本語をレプチャ語や朝鮮語と結び付ける邪説、\citeA{吉田91} 
のような、日本語をあらゆるものの起源にしてしまう暴説などがあとを断たな
いが、これらに対しては\citeA{安本91} による学問的な見地からの反論にゆ
ずるとして、本論文で私が主張することはこれらの先行非研究とは一線を画す
ものであることを強調しておきたい。その意味では、英語の起源を日本語に置
く真に学問的な先行研究は存在しないと言っても過言ではない。


\clearpage
\chapter{語呂あわせとの相違}

本稿の初期の版に対して、次のような批判がよせられたことがある。

\begin{quote}
    吉原源三郎とかいう素人のたわごと「英語起源日本語説』なるものは、
    犬小屋をケンネル (kennel)、辞書を字引く書なり (dictionary) と言
    うたぐいの与太にすぎない \cite{山崎91a}。
\end{quote}

  一体彼はどうして私の説を、犬小屋をケンネル(犬寝る)と覚えるような語
呂あわせと同じものだと思ってしまったのであろう。試みに私が例示した、日
本語から英語になったと思われる単語のいくつかを引出してみれば、すぐさま
それが語呂あわせの英単語記憶法などとは全く違うものであることが理解でき
るではないか。たとえば、

\eenumsentence{
\item 坊や (boya) → boy (少年)
\item 名前 (namae) → name (名前)
}
\noindent
これらのどこが字引く書なり、と同じだというのであろう。

  boya なる単語が、母音の弱化により boy に変じていくという仮説は、山崎
氏がその代表的著作\citeA{山崎69} の中で展開した学説と全く同じではないか。

  だから、氏が私の論理を与太だと決めつけることは、そのまま自分の論理を
否定することになるのである。彼には自分が何を言っているのかわかっていな
いのだろうか。

  いずれにしても、山崎氏の私への反論は、私のあげた例証のうち少々言語伝
達の構造が複雑なものについて集中していることは明らかである。

  そのくせ、氏は私のあげる例のうち、あまりにも単純なもの、ストレートに
意味が伝達されているもののことについては口をふさいで語ろうとしない。す
なわち、反論の余地がないのである。次の例を参照されたい。

\enumsentence{
\begin{tabular}[t]{lcl}
    負う (ou)       & → & owe (負う)\\
    たぐる (taguru) & → & tag(引き寄せる)\\
    疾苦 (sikku)    & → & sick(病気)
\end{tabular}
}
\noindent
これらの例は、ケチのつけようがないほど完全に言語伝達されている。そこで
山崎氏はこういう沢山の例を無視するわけである。

  それならば、次の例についてどうして氏は反論しないのであろう。これはな
かなか伝達の構造が複雑なのであるのに。

\enumsentence{
場取る (batoru) → battle (戦い)
}
\noindent
この例は、日本語が英語のもとになったことを見事に証明してくれるものなの
である。

  すなわち、欧米では戦いとは、敵の王を殺すことであったのに対し、日本で
は、敵の領地を奪うこと、つまり場を取る(盗る)ことであったという文化的
差異がこの一語にはこめられている。

  さらに、山崎氏は、次のような反論をしている。

\begin{quote}
    この吉原源三郎説の最も弱いところは、比較される二つの言葉に意味
    の共通性がないことが非常に多いことである。たとえば彼は、日本語
    の(掘る)という動詞を、hole (穴)という英語の名詞と対比して
    みせる。掘るから穴だというこの漫才のような対比は、学問的立場か
    らは到底認められないものである。cold (冷たい)を(凍るぞ)と
    やってみせるのも同様のおちゃらけである \cite[96]{山崎91b}。
\end{quote}
\noindent
彼の頭の中にある言語の伝達は、二国間でひとつの言葉が全く同じ意味に伝わ
る、ということであるらしい。それ以外のものは、たとえどんなに共通性が強
くても、無いのと同じ、なのである。

  英語の hole が、掘る、という意味ならば認めるが、そうでない以上そんな
ものはおちゃらけだ、と言い張る彼は、本当に言語学者なのだろうか。言語と
いうものが、いかに誤って伝達され、時と共にその意味すら変化していくとい
うことを知らない言語学者がどうして存在しているのか、私には謎である。

  たとえば、「前」という、位置、方向を表す言葉が、「お前」となって二人
称の人物を表す言葉になり、更に下ってそれが尊称から蔑称に変化した、とい
うような事実は、言語学者山崎にとっては、無いのと同じ、なのであろうか。
もしそうだとしたらそんな学者を認めておくわけにはいかない。

  私の言うところの(伝日本語人)が、地面に穴をあけてこと作業を「掘る」
というのだと教えた時、(源英語人)がその言葉の意味を、そこにあいた穴だ
と解釈したというのは非常にありうべき事柄である。そこに漫才師の登場する
余地はほとんどない。

  同様に、「この寒さでは、明日は水が凍るだろう」と言った時、k\^oru → 
cold、コールドが冷たいという意味に受け止められたことは誰のおちゃらけで
もない。同じような考察によって、私は堂々と次のような類似の例も提出して
いるのである。

\eenumsentence{
\item 抛る (h\^oru) → fall (落ちる)
\item 述べる (noberu) → novel (小説)
}


\clearpage
\chapter{実験結果}
TSOMを用いて行った解析をいかに示す


\clearpage
\chapter{おわりに}

本来は、どのようにして日本語が海を渡り、英語の起源になったかに関する仮
説を本論文に含める予定であったが、この問題に関して本格的に論じるには別
の大部の書を必要とするし、その作業は既に私の手によってなされているため
に、割愛せざるを得なかった(拙著\citeA{吉原93} を参照されたい)。


\clearpage



\bibliographystyle{jtheapa}
\bibliography{thesis}

\appendix
\chapter{おまけ}
おまけ

\end{document}

%コンパイル終わってからのターミナルで、このフォルダまで来て、pbibtex thesisj を実行しないと参考文献が反映されない
