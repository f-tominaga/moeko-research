\chapter{語呂あわせとの相違}

本稿の初期の版に対して、次のような批判がよせられたことがある。

\begin{quote}
    吉原源三郎とかいう素人のたわごと「英語起源日本語説』なるものは、
    犬小屋をケンネル (kennel)、辞書を字引く書なり (dictionary) と言
    うたぐいの与太にすぎない \cite{山崎91a}。
\end{quote}

  一体彼はどうして私の説を、犬小屋をケンネル(犬寝る)と覚えるような語
呂あわせと同じものだと思ってしまったのであろう。試みに私が例示した、日
本語から英語になったと思われる単語のいくつかを引出してみれば、すぐさま
それが語呂あわせの英単語記憶法などとは全く違うものであることが理解でき
るではないか。たとえば、

\eenumsentence{
\item 坊や (boya) → boy (少年)
\item 名前 (namae) → name (名前)
}
\noindent
これらのどこが字引く書なり、と同じだというのであろう。

  boya なる単語が、母音の弱化により boy に変じていくという仮説は、山崎
氏がその代表的著作\citeA{山崎69} の中で展開した学説と全く同じではないか。

  だから、氏が私の論理を与太だと決めつけることは、そのまま自分の論理を
否定することになるのである。彼には自分が何を言っているのかわかっていな
いのだろうか。

  いずれにしても、山崎氏の私への反論は、私のあげた例証のうち少々言語伝
達の構造が複雑なものについて集中していることは明らかである。

  そのくせ、氏は私のあげる例のうち、あまりにも単純なもの、ストレートに
意味が伝達されているもののことについては口をふさいで語ろうとしない。す
なわち、反論の余地がないのである。次の例を参照されたい。

\enumsentence{
\begin{tabular}[t]{lcl}
    負う (ou)       & → & owe (負う)\\
    たぐる (taguru) & → & tag(引き寄せる)\\
    疾苦 (sikku)    & → & sick(病気)
\end{tabular}
}
\noindent
これらの例は、ケチのつけようがないほど完全に言語伝達されている。そこで
山崎氏はこういう沢山の例を無視するわけである。

  それならば、次の例についてどうして氏は反論しないのであろう。これはな
かなか伝達の構造が複雑なのであるのに。

\enumsentence{
場取る (batoru) → battle (戦い)
}
\noindent
この例は、日本語が英語のもとになったことを見事に証明してくれるものなの
である。

  すなわち、欧米では戦いとは、敵の王を殺すことであったのに対し、日本で
は、敵の領地を奪うこと、つまり場を取る(盗る)ことであったという文化的
差異がこの一語にはこめられている。

  さらに、山崎氏は、次のような反論をしている。

\begin{quote}
    この吉原源三郎説の最も弱いところは、比較される二つの言葉に意味
    の共通性がないことが非常に多いことである。たとえば彼は、日本語
    の(掘る)という動詞を、hole (穴)という英語の名詞と対比して
    みせる。掘るから穴だというこの漫才のような対比は、学問的立場か
    らは到底認められないものである。cold (冷たい)を(凍るぞ)と
    やってみせるのも同様のおちゃらけである \cite[96]{山崎91b}。
\end{quote}
\noindent
彼の頭の中にある言語の伝達は、二国間でひとつの言葉が全く同じ意味に伝わ
る、ということであるらしい。それ以外のものは、たとえどんなに共通性が強
くても、無いのと同じ、なのである。

  英語の hole が、掘る、という意味ならば認めるが、そうでない以上そんな
ものはおちゃらけだ、と言い張る彼は、本当に言語学者なのだろうか。言語と
いうものが、いかに誤って伝達され、時と共にその意味すら変化していくとい
うことを知らない言語学者がどうして存在しているのか、私には謎である。

  たとえば、「前」という、位置、方向を表す言葉が、「お前」となって二人
称の人物を表す言葉になり、更に下ってそれが尊称から蔑称に変化した、とい
うような事実は、言語学者山崎にとっては、無いのと同じ、なのであろうか。
もしそうだとしたらそんな学者を認めておくわけにはいかない。

  私の言うところの(伝日本語人)が、地面に穴をあけてこと作業を「掘る」
というのだと教えた時、(源英語人)がその言葉の意味を、そこにあいた穴だ
と解釈したというのは非常にありうべき事柄である。そこに漫才師の登場する
余地はほとんどない。

  同様に、「この寒さでは、明日は水が凍るだろう」と言った時、k\^oru → 
cold、コールドが冷たいという意味に受け止められたことは誰のおちゃらけで
もない。同じような考察によって、私は堂々と次のような類似の例も提出して
いるのである。

\eenumsentence{
\item 抛る (h\^oru) → fall (落ちる)
\item 述べる (noberu) → novel (小説)
}

