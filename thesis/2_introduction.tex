\chapter{先行研究}

\section{英語の起源}

私がここで論証しようとしていることは次の短い一文に要約できる。

\enumsentence{
英語の起源は日本語である。
}

  しかし、内容の大きさが文の短さとは比例しないことは、言を俟たない。思
えば、従来どの比較言語学者も、日本語が何か他の言語の起源であるというよ
うな発想を持ち得なかったのである。

  中学一年の時だったと思う。試験の時、次の言葉を英語にせよ、という問題
が出て、その中に<名前>というのがあった。

  私の隣の席の少年は、その<名前>を意味する英単語が思い出せなかった。
それで彼は、とにかく何か書いておこうという気持から、答えの欄にローマ字
で namae と書いた。あとで答案用紙を返してもらう時、彼は悪ふざけをする
なと先生に叱られるかも知れないと心配したが、その心配は無用だった。彼の
答の namae は、正解と比べて a がひとつ多かっただけで、先生は彼が苦肉の
策としてローマ字を書いたということを見抜けなかったのである。

\enumsentence{
name (ネーム)と namae (名前)
}
\noindent
この時彼の言った言葉、「英語では名前を名めー、というのか」を思い出した
時、私の頭に電撃的にひらめくものがあった。

  name と namae の類似性は、偶然というにはあまりにも近すぎる。name の
語源は namae ではないのか、と想像せざるを得ないではないか。

  研究に値する発見であるように、思考の柔軟な私には思えた。これが私の学
説の出発点となったのである。

  日英語単語間に、ほかに類似性の見られるものはないだろうかという探索の
作業が積み重ねられた。その結果、ひとつの新学説を成すに十分と思われる
292例もが発見されたのである。\footnote{
%--------
先の脚注で触れたように、その後の研究の進展により、英語の起源が日本語
であることを立証する例は326になった。
} %--------

  ここにその中から2, 3 の例を引けば次のようになる。

\eenumsentence{
\item 汁 (ju) → juice (汁)
\item 斬る (kiru) → kill (殺す)
\item  だるい (durui) → dull (鈍い)
}

  これだけの例を見ただけでも、日本語と英語との間には否定し難い関連があ
ることが感じられるであろう。ましてや、292例のすべてに目を通し終えた時、
思考の柔軟な読者には、英語と日本語には何か根本的な共通性があることを確
信せざるを得ないであろう。\footnote{
%--------
もちろん、次のような反論が即座に返ってくることは予想できる。

\eenumsentence[i]{
\item 文法が違う。
\item 日本語は膠着語で英語は屈折語である。
\item いつ、どのように伝わったのかはっきりしない。
}
\noindent
これらの問題は将来の研究の進展に委ねたいが、本能的に二つの言語が兄弟で
あることは直感できるはずである。常識という思考力の枷に毒された三流学者
でない限りは。
} %--------

\section{従来の研究の問題点}

  日本語と他の言語の共通性を指摘する研究はもちろん私の研究が最初のもの
ではない。古くは、明治時代に木村鷹太郎氏が、日本人の先祖はエジプト人で
あり、ギリシャ人でありローマ人であるという大胆な説を主張し日本話とギリ
シャ語・ラテン話・英語との類似を指摘している \cite{木村81}。木村によると
次のような英語の単語は日本語と起源が同じだという。

\enumsentence{
\begin{tabular}[t]{*{9}{l|}l}
    夕べ&ダメ  &君  &籠  &なんぼ&潮 &骨  &ソロリ&身&百合\\
\hline
    eve &damage&king&cage&number&see&bone&slowly&me&lily
\end{tabular}
}
\noindent
一目瞭然であるが、これは単なる英単語の駄洒落による暗記法であり、学問的
な裏付けを欠いている。

  最近に至るまで、このような胡散臭い研究は後を断たないが、英語圏からの
研究もある、\citeA{スミサナ92} によると、日本人の先祖はアメリカ・インディ
アンであり、インディアン語に由来するアメリカの地名・人名・部族名はすべ
て日本語で解読できるという。例えば、次のような例が「証拠」としてあげら
れている。

\enumsentence{
\addtolength{\tabcolsep}{-0.3em}
\begin{tabular}[t]{@{}l|l||l|l||l|l}
テキサス & 敵刺す & ミズーリ & 水入り江 & マサチューセッツ & 鱒駐節\\
ミシシッピー & 水疾飛 & ワイオミング & 上の民家 & オクラホマ & 遅れ本真\\
オハイオ & おはよう & カンザス & 関西 & ケンタッキー & 関東京\\
メキシコ & 茅始処 & カナダ & 金田 & ナイアガラ & 荷揚げ場\\
アパッチ & あっぱれな者 & エスキモー & アシカの肝 & ジェロニモ & 地浪人者
\end{tabular}
}
\noindent
これではまるで暴走族の名ではないか。訳者あとがきによれば、著者は「最近
の日英辞典」と使ったとあるが、ケンタッキーは明治時代になってからできた
のであろうか。

  他にも、\citeA{安田55}、\citeA{藤村89}、\citeA{李89} などの、万葉集
を根拠にして日本語をレプチャ語や朝鮮語と結び付ける邪説、\citeA{吉田91} 
のような、日本語をあらゆるものの起源にしてしまう暴説などがあとを断たな
いが、これらに対しては\citeA{安本91} による学問的な見地からの反論にゆ
ずるとして、本論文で私が主張することはこれらの先行非研究とは一線を画す
ものであることを強調しておきたい。その意味では、英語の起源を日本語に置
く真に学問的な先行研究は存在しないと言っても過言ではない。

